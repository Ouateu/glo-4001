\documentclass[12pt]{article}
\usepackage{epsfig}
\usepackage{graphicx}
\usepackage{color}
\usepackage[frenchb]{babel}
\usepackage{subfigure}
\usepackage{amsmath}
\usepackage{amssymb}
\usepackage{enumerate}

% Pour pouvoir utiliser les accents directement dans LaTeX, sans utiliser les commandes \'
%\usepackage[latin1]{inputenc} % entree 8 bits iso-latin1
\usepackage[utf8]{inputenc} % entree 8 bits utf8, fonctionne avec MikTeX sur Windows.
\usepackage[T1]{fontenc}      % encodage 8 bits des fontes utilisees

% Pour agrandir les marges
\addtolength{\oddsidemargin}{-.875in}
\addtolength{\evensidemargin}{-.875in}
\addtolength{\textwidth}{1.75in}
\addtolength{\topmargin}{-.875in}
\addtolength{\textheight}{1.75in}

\begin{document}
\selectlanguage{french}
\title{Introduction à la robotique mobile : travail pratique 3  \\  Date de remise : 8 décembre 2017, 23h55.}
\author{Instructeur : Philippe Giguère}

\maketitle

\section {Solution par filtre à particules (10 pts)}


\section{Filtre Kalman étendu (EKF) (40 pts pour GLO-4001, 35 pts pour GLO-7021)}
\label{EKF}

\subsection{Matrices Jacobiennes (10 pts)}

\subsubsection{Détermination de $\Gamma$}
Soit nous définissons $\Gamma$ comme la matrice de dérivé des fonctions de mesure selon la commande:
\begin{equation}
\Gamma =
\begin{bmatrix}
    0   \\
    \\
    \dfrac{\exp(-u/2)}{(1+\exp(-u/2))^2} \\
\end{bmatrix}
\end{equation}

\subsubsection{Détermination de $\Lambda$}
\begin{equation}
\Lambda =
\begin{bmatrix}
    \frac{-0.07}{d^2}   \\
    0
\end{bmatrix}
\end{equation}

\subsection{Implémentation du filtre EKF (30 pts pour GLO-4001, 25 pts pour GLO-7021)}
\begin{enumerate}[a)]
\item Vous connaissez exactement la position initiale du robot au départ. Vous initialisez donc la matrice d'état à $X=[d_{init} \mbox{ }0]^T$, et la matrice de covariance à
$$
P=
\begin{bmatrix}
0 &0\\
0 & 0
\end{bmatrix}
$$

\item Vous avez une bonne idée de la position initiale du robot au départ, mais vous n'êtes pas confiant à $100~\%$. Vous initialisez donc la matrice d'état à $X=[d_{init} \mbox{ }0]^T$, et la matrice de covariance à
$$
P=
\begin{bmatrix}
25 &0\\
0 & 1
\end{bmatrix}
$$

\item Vous croyez connaitre exactement la position initiale du robot au départ, mais cette valeur est, dans les faits, erronée. Vous initialisez donc la matrice d'état à $X=[(d_{init}+1) \mbox{  } 0]^T$, et la matrice de covariance à
$$
P=
\begin{bmatrix}
0 &0\\
0 & 0
\end{bmatrix}
$$

\item Vous n'êtes pas sûr que $d=d_{init}+1$ est la bonne position de départ du chariot. Vous initialisez donc la matrice d'état à $X=[(d_{init}+1)  \mbox{  }  0]^T$, et la matrice de covariance reflète cette incertitude car vous l'initialisez à:
$$
P=
\begin{bmatrix}
100 &0\\
0 & 1
\end{bmatrix}
$$

\end{enumerate}

\section {Localisation globale par filtre à particules (35 pts)}

\subsection{Cas 1 : l'angle est connu (GLO-4001 : 30 pts, GLO-7021 : 20 pts)}

\begin{itemize}
\item une justification de votre choix $\sigma_{angle}$;
\item les divers paramètres utilisés dans le filtre;
\item description qualitative de la distribution des particules au fil du temps.
\end{itemize}

\end{document}
